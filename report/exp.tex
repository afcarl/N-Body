% File: exp.tex
% Date: Fri Aug 31 00:53:05 2012 +0800
% Author: Yuxin Wu <ppwwyyxxc@gmail.com>
\section{Summary \& Experience}
	在第二次作业中,我熟悉了gtkmm的开发,但由于清华FIT楼集群计算机没有gtkmm相关库,在第三次作业中我便使用了Xlib.
	Xlib功能太弱,gtk所支持的信号功能对于这次的程序很重要,因此此次我使用gtk.
	
	在并行方面,由于N体问题每个计算都涉及很多其他数据的参与,
	可以独立并行计算的部分不多,因此对并行算法的设计中不得不在效率与准确度之间做一些取舍.
	因此大家的程序效率,演示效果都有不少差别.如我的MPI程序,就因计算时的几次数据同步耗去了大量时间.

	上次的作业中,算法已经给出,只需设计并行部分.而这次的作业中算法是个难点,较好的选择常数,处理碰撞,
	使得demo中不出现不科学的运动模式花了我大多数的时间.

	这次作业模拟的是物理问题,引力与碰撞的相关问题也让我温习了物理知识,设计常量的过程也让我感受到了宇宙常数的难以选择,不合适的常数总使得物体的运动十分不规则,
	让我更深刻的体会到了关于神秘宇宙常数的``Fine-tuned Universe''的论断\cite{tune}.
